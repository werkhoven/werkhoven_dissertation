% set margins and font
\documentclass[12pt,letterpaper]{article}
\usepackage[margin=1in]{geometry}
\usepackage{helvet}
\renewcommand{\familydefault}{\sfdefault}

% amsmath and amssymb packages, useful for mathematical formulas and symbols
\usepackage{amsmath,amssymb}

% Use Unicode characters when possible
\usepackage[utf8x]{inputenc}

% array package and thick rules for tables
\usepackage{array}

% Remove comment for double spacing
\usepackage{setspace} 
\doublespacing

% text symbol package
\usepackage{textcomp}

\begin{document}
\vspace*{0.2in}

% begin roman numeral page numbering
\pagenumbering{roman}

% TITLE PAGE
\begin{titlepage}
	\centering
	\vspace{1cm}
	{\scshape \large The Structure of Behavioral Variation Within a Genotype\par}
	\vspace{1.0cm}
	A dissertation presented\par \vspace{0.35cm}
	by\par \vspace{0.35cm}
	Zachary Werkhoven\par \vspace{0.35cm}
	to\par \vspace{0.35cm}
	The Department of Molecular and Cellular Biology\par \vspace{0.35cm}
	in partial fulfillment of the requirements\par \vspace{0.35cm}
	for the degree of\par \vspace{0.35cm}
	Doctor of Philosophy\par \vspace{0.35cm}
	in the subject of\par \vspace{0.35cm}
	Molecular and Cellular Biology\par \vspace{0.35cm}
	\vfill
	Harvard University\par
	Cambridge, Massachusetts\par
	August 2019
	\vfill
\end{titlepage}
\clearpage


% COPYRIGHT PAGE
\topskip0pt
\vspace*{\fill}
    \begin{center}
    \textcopyright \hspace{0.01in} 2019 Zachary Werkhoven
    \end{center}
    \begin{center}
    All Rights Reserved.
    \end{center}
    \thispagestyle{empty}       % set page numbering style to blank
\vspace*{\fill}
\clearpage

% ABSTRACT
\setcounter{page}{3}
\begin{abstract}
    aasdfj addsa sd ds das s adfsd adsf aasdfj addsa sd ds das s adfsd adsf aasdfj addsa sd ds das s adfsd adsf aasdfj addsa sd ds das s adfsd adsf aasdfj addsa sd ds das s adfsd adsf aasdfj addsa sd ds das s adfsd adsf aasdfj addsa sd ds das s adfsd adsf aasdfj addsa sd ds das s adfsd adsf aasdfj addsa sd ds das s adfsd adsf aasdfj addsa sd ds das s adfsd adsf aasdfj addsa sd ds das s adfsd adsf 
\end{abstract}

% insert chapter title page
\clearpage
\begin{center}
    \Large\section{Chapter 1}
    \pagenumbering{arabic}      % switch page numbering to arabic numerals
    \setcounter{page}{1}        % reset the page counter to 1
    \thispagestyle{empty}       % set page numbering style to blank
    \clearpage
\end{center}

\subsection{Introduction}

Individuals display idiosyncratic differences in behavior that often persist through time and are robust to situational context. Some persistent individual behavioral traits commonly occur in correlated groups and can therefore be said to covary together. Observations covarying behavioral traits have been limited in scope and thus the sources and extent of behavioral covariation is not well-understood. Behavioral covariation occurs any time behavioral traits are linked and can be detected as a correlation between two or more behaviors. For example, navigation with respect to light (phototaxis) and navigation with respect to temperature (thermotaxis) covary, then individuals that are positively phototactic are more likely to be positively thermotactic and vice versa. 

Behavioral ethologists have long understood that ”types” of human and animal personalities can be described as groups correlated that occur together. Propensities for human social behaviors such assertiveness, talkativeness, and impulsiveness are commonly thought to occur together and described as extraversion, while passivity, shyness, and deliberateness are described as introversion. Although the total space of human social behaviors is very high dimensional, extraversion-introversion is thought to constitute a fundamental dimension in human social behavior, thereby reducing the effective dimensionality of the space of social behaviors. Behavioral symptoms also frequently correlate when neuronal function is perturbed in human mental disorders. Severity of stereotyped motor behaviors and language impairments correlate in autism spectrum disorder, and sleep correlates with repetitive behaviors in bipolar disorder. In animal species, correlated suites of behaviors are known as behavioral syndromes. Aggressive behaviors such fighting over mates or food are correlated with exploratory behaviors such as foraging and social interaction in diverse species of insects, arachnids, fish, and birds. These behaviors are frequently described as falling on an axis of boldness-shyness. 

The prevalence of behavioral correlation suggests that covariation is likely a broad feature of behavior, but the extent to which behaviors covary, how behavioral covariation arises, and whether or not behavioral variation has a characteristic structure is not well-understood. Profiling behavioral variability therefore has the potential to answer important questions in biology: Is effective dimensionality of behavioral high or low? What does the structure of behavioral variability look like across individuals and across genotypes? Is the correlational structure of behavior stereotyped or variable? Can behavioral correlations be linked to neural circuit structure or gene expression?

\subsection{Drosophila as a Model of Individual Behavioral Variation}

Organisms such as insects with significantly less complex brains than humans (approximately 6 orders of magnitude fewer neurons) display a diverse array of behaviors that vary by genotype and environment. Fruit flies are an attractive model for studying behavioral variation. In addition to the ease of maintaining many thousands of individuals and wealth of genetic tools available in fruit flies, they respond to a variety of visual, olfactory, thermal and mechanical cues. Moreover, fruit flies show considerable behavioral variation across mutant and wild-type genotypes.

Assaying individual behavioral variability poses many practical challenges. Capturing the effective dimensionality of individual variation requires profiling individual behaviors very broadly. Measuring behavioral covariability broadly comes at the cost increased number of comparisons as the number of pair-wise combinations of measures grows at a factorial rate ${n \choose 2}$. Even with a relatively small number of behavioral measurements (e.g. tens of measures), high statistical power is required to deal with problems of multiple comparisons. Small body size, short generation time, and high fecundity make flies well-suited to assaying individual behaviors at high throughput. Thus drosophila has emerged as a model for individual behavioral variation. 

Fruit flies also display individual variability in locomotor handedness that persists well into adulthood. Fly handedness can be measured in flies by placing them in Y-shaped arenas scoring their trajectories as right or left-handed as they pass through the center of the arena, choosing either the right or left arm. Hundreds of flies can be assayed simultaneously this way making hundreds of choices each, providing robust measurements of individual with high statistical confidence. When assayed this way, individuals display significant individual variation (e.g. some flies display extreme right turn probabilities of 0.9 and 0.1). Similarly high-throughput assays have been used to measure individual variation in phototactic and thermotactic preferences. In all behaviors measured, individual flies showed individual variability well beyond what would be expected if individual choices were drawn randomly from a shared distribution (i.e. if all individuals make choices with the same probability). These findings demonstrate the feasibility of drosphila as a model system for assaying individual behavioral variability and show that flies display individual variability in behaviors with and without apparent ethological consequence. 

\subsection{Neural Circuit Bottlenecks as a Potential Source of Behavioral Covariation}

Isogenic D. melanogaster raised in the same environment show idiosyncratic variations in many
behaviors across sensory modalities that are stable throughout an individual’s life 6-8. Interestingly, inbred flies show individual behavioral variability in locomotor handedness. Counter-intuitively, variability in this context increases as genetic diversity decreases. Crossing flies from the distribution tails of locomotor handedness together (i.e. extreme "righties" crossed to each other and extreme "lefties" crossed to each other). Collectively these results suggest a role for non-genetic, non-heritable sources of individual behavioral variability.

While there are well-characterized and understood examples of genetic and developmental constraints on trait covariation, evidence of behavioral covariation while genes and environment are held constant suggests that neural architecture is a likely developmental source of behavioral constraint. Behavioral circuits that depend on overlapping sets of neurons may vary together in ways that reflect variations in those neurons. For example, flies with a higher or lower ratio of inhibitory to excitatory synapses in the descending motor neurons of their rear legs could plausibly show correlations in many posterior limb behaviors such as wing and abdomen grooming. If such circuit bottlenecks induce behavioral covariation, principal behavioral dimensions may be mappable to locations in the nervous system. Motor neurons are an obvious site of convergence in behavioral circuits, but more elusive sites of convergence may exist in the web of connections that underlie sensorimotor transformations such as the central brain.

Neural circuit constraints are one interesting possible source of behavioral covariation and one that may help explain how variation arises in the absence of genetic diversity but remain a largely unexplored source of constraint on behavior due to practical considerations associated with measuring it independently of genetic and environmental sources. Isogenic and environmentally matched fruit flies offer a system to study individual differences in neural circuitry as a source of behavioral variation. In particular, large behavioral screens that can holistically profile behavior may help identify clusters of behaviors that correlation across individuals. When combined with sparse genetic driver lines thermogenetic and optogenetic tools for neural activation and silencing provide a method to systematically perturb small components of neural circuits. Combining behavioral screens with neural circuit manipulations offers two possible benefits: 1) the potential to independently validate clustered behaviors by inducing correlated shifts in behavior through circuit manipulation and 2) localization of behavioral correlations to specific neurons or brain regions. Probing neural circuit constraints on behavior is a necessary step in elucidating the interplay between genes and the underlying neural structure in both evolution and disease, and may suggest a mechanisms for pleiotropic effects on behaviors.

\subsection{Methods of Holistic Behavioral Profiling}

The above example of phototaxis and thermotaxis are behaviors one might expect to correlate because sources of light such as the sun are also sources of heat. Although it seems probable that many clusters of correlated behaviors may belong to groups behaviors of shared ethological and physical relevance, correlated behaviors need not have any intuitive relationship. Known correlated groups of behaviors such as boldness-shyness were discovered because specific hypothesis about related behaviors led to studies that were designed to specifically assay those behaviors, but one compelling reason to profile behavioral as holistically as possible is to reveal surprising behavioral clusters that may point to interesting features of the behaviors themselves or an expected and shared biological underpinning. Probing the structure of behavioral variability more generally requires measuring as many behaviors as possible to capture the totality of behavioral variation as completely as possible.

Profiling individual as deeply as possible across a broad range of contexts has become a recent goal of behavioral ethologists. Unsurprisingly, animal behaviors are very diverse and capturing their entire behavioral repertoire presents substantial challenges such as: how to define behaviors, how to feasibly measure multiple behaviors in the same animal, how to measure behaviors that are broadly representative of an animal's total behavior, and how to achieve high enough sampling.

Historically, behavioral measurements have been scored by hand. The limited throughput of manual scoring forced a trade off between depth of and breadth of measurements, resulting in two generically broad approaches to profiling behaviors. The first approach typically focuses on small numbers of individuals where their behaviors are scored on a trial by trial basis for each individual separately and are therefore typically low throughput. The ability to focus on single trials sequentially allows for complex behavioral tasks with potential for minute discrimination between responses which may be composed of multiple features (though the approach also accommodates simple paradigms as well). Examples of this approach include psycho-physical measurements in humans and animals or manual scoring of social interaction between individuals. The second approach, focuses on collective measurement of the responses of groups and is therefore potentially high-throughput. The need to score behavior in groups destroys information about individual responses and also commonly requires thresholding or binning responses. Examples of this approach include classic assays of phototaxis and geotaxis in fruit flies where population responses are measured as the fraction of animals falling inside a number vertically marked zones after being tapped to the bottom of a tube.

More recently, omics based approaches to behavioral measurement have been applied to the study of behavior with success. The Drosophila Genomics Reference Panel (DGRP) is one example of such an approach. The DGRP is a collection of approximately 200 \textit{Drosophila melanogaster} lines of inbred from the wild population of flies in Raleigh, North Carolina. The DGRP serves as snapshot of the genetic diversity present in a wild population of flies and a common reference point between studies. This common reference point has enabled the construction of a database of behavioral phenotypes measured on the DGRP across more than a dozen studies. Behaviors assayed on the DGRP are largely measured at the population-level and do not contain information about individual behaviors. Nonetheless, the DGRP phenotype collection offers a unique opportunity to study behavioral covariation across genotypes.

Machine learning can be applied to automated measurements of behavior for representations of behavior that are both nuanced and broad in scope. The Janelia Automated Animal Behavior Annotator (JAABA) uses supervised machine learning to classify animal behaviors from video data. Tracking data extracted from videos such as the position and orientation of animals and their body parts is used to engineer a variety of features that contain information about their behavior changes over time. Users then define a set of behaviors to classify and provide examples of each. Classifiers are trained examples and then used to detect new instances of the behaviors in videos that the classifiers have never seen. The rich set of features fed into the algorithm contains detailed information about the posture and movement of the animals that enables the classifiers to detect minute details of behavior that might be difficult for a human to detect. 

JAABA is of particular interest because it has been used extensively to classify behavior of individual fruit flies. Robie et al. used JAABA to screen the effects of neural circuit manipulation across a diverse set of neural driver lines. In total, they profiled the behavior of 2,204 Gal4 lines driving expression a temperature-sensitive cation channel, dTrpA1. JAABA was used to classify 14 behaviors on short videos captured of groups of flies at the restrictive temperature. Although measurements were performed on individuals, the individual component of variability cannot be decoupled from inter-individual internal state variation the short duration of the videos and lack of repeated measurements of the individuals. The study represents a wealth of behavioral variability in response to neural circuit perturbation.

Unlike supervised learning methods which require behaviors to be defined before classification, unsupervised classification of behavior can be used to assign labels to behaviors without any a priori definition. Unsupervised classification has recently used to characterize fruit fly behavior via the motion-mapper pipeline developed by Berman et al. The motion-mapper pipeline relies on the assumption that animal behaviors consist of stereotyped, spatio-temporal postural sequences to generate distinct behavioral labels. The pipeline assays spontaneous exploratory behavior via videos of single flies recorded at high spatial and temporal resolution. Pixels from images are decomposed into high-dimensional representation of the animal's postural dynamics. Ultimately, frames in the high-dimensional representation are clustered in a low-dimensional representation via the t-SNE algorithm and assigned a classification based on those clusters. Unsupervised classification is particularly promising for holistic profiling of individual behavioral because 1) it attempts represent the total observed behavioral space by assigning a label to all frames and 2) it requires no assumption about the number of distinct classifications.

The motion-mapper pipeline has been applied to natural behavioral variation and the effects of neural circuit perturbations on behavior. Berman


\subsection{Evolution of Correlated Behaviors and the G-Matrix}

Do behaviors evolve independently or as sets of behavior? 
The introduction of covariation in behavior lowers the effective dimensionality of behaviors that animals can exhibit. This raises the possibility that covariation may constrain the space of evolutionary space that behaviors can explore, causing some behaviors to evolve as sets of behaviors. This possibility is analogous to the phenomenon of pleiotropy where a single gene influences more than one phenotype, which has been hypothesized to have a stabilizing effect on evolution. Evolutionary geneticists typically think of correlated traits as being coupled by mutual dependence on common genes, developmental origins, or environmental factors. Regardless of the source, the quantitative study of covariation of traits and their effect of evolutionary trajectories has been formalized in the G-matrix (i.e. the matrix of additive genetic variance and covariances). 

There are two basic questions of interest to behavioral ethologists and evolutionary biologists: 1) Which behaviors are correlated or anti correlated? 2) Is the correlational structure of behaviors stable or unstable? Understanding which behaviors are correlated and how consistently groups of behaviors


% insert chapter title on its own page
\clearpage
\begin{center}
    \Large\textbf{Chapter 2}
    \thispagestyle{empty}       % set page numbering style to blank
    \clearpage
\end{center}

\section*{Chapter 2 Title}

% Please keep the abstract below 300 words
\section*{Abstract}

Fast object tracking in real time allows convenient tracking of very large numbers of animals and closed-loop experiments that control stimuli for multiple animals in parallel. We developed MARGO, a real-time animal tracking suite for custom behavioral experiments. We demonstrated that MARGO can rapidly and accurately track large numbers of animals in parallel over very long timescales. We incorporated control of peripheral hardware, and implemented a flexible software architecture for defining new experimental routines. These features enable closed-loop delivery of stimuli to many individuals simultaneously. We highlight MARGO's ability to coordinate tracking and hardware control with two custom behavioral assays (measuring phototaxis and optomotor response) and one optogenetic operant conditioning assay. There are currently several open source animal trackers. MARGO’s strengths are 1) robustness, 2) high throughput, 3) flexible control of hardware and 4) real-time closed-loop control of sensory and optogenetic stimuli, all of which are optimized for large-scale experimentation.

\linenumbers

% Use "Eq" instead of "Equation" for equation citations.
\section*{Introduction}

The recent introduction of automated methods for measuring behaviors offers a level of throughput that has gradually reduced the need to trade between depth and breadth of behavioral measurements. Cheap electronic components such as infrared beam break detectors enabled course detection of animal position. The TriKinetics Drosophila Activity Monitor (DAM) estimates animal activity level by counting the number crosses over a beam break sensor. A similar strategy has also been used to measure phototactic behavior in a T-maze by positioning sensors in arms on either side of the choice point. More recently, the availability of inexpensive, high-resolution cameras has enabled widespread use of video data in animal behavioral measurement. Automated tracking of animal centroids and body segments have revolutionized the study of behavior by enabling precise measurement of the location and posture of animals. When coupled with other animals or automated stimulus delivery, animal tracking provides detailed descriptions of behavioral dynamics in response social and stimulus driven contexts. 

Automated animal tracking methods have become commonplace in the study of behavior. They enable large sample sizes, high statistical power, and more rapid inference of mechanisms giving rise to behavior. Existing animal trackers vary in computational complexity and are often specialized for particular imaging configurations or behavioral measurements. Trackers can assist in a wide range of experimental tasks such as monitoring activity, measuring response to stimuli \cite{Fry_TrackFly_2008,Donelson_High_2012}, and locating body parts over time \cite{Mathis_DeepLabCut_2018,Pereira_Fast_2018}. Some trackers are designed to track and maintain identities of multiple individuals occupying the same arena \cite{Prez-Escudero_idTracker_2014,Eyjolfsdottir_Detecting_2014,Rodriguez_ToxId_2017,romero-ferrero_2019} while others measure the collective activity of groups without maintaining identities or rely on physical segregation of animals to ensure trajectories never collide \cite{Ramot_The_2008,Swierczek_High_2011,Itskovits_A_2017,Scaplen_Automated_2019}. But few of these trackers are designed as platforms for high throughput, hardware control, and flexible experimental reconfiguration. 

Improvements in machine learning and template matching approaches to object localization and classification have made it possible to efficiently train models that accurately track and classify a variety of animal species and visually distinguish identities of individuals across time \cite{Eyjolfsdottir_Detecting_2014,Prez-Escudero_idTracker_2014,schneider_2018,romero-ferrero_2019}. Tracking individual identity in groups requires resolving identities through collisions where bodies are overlapping. FlyTracker and idTracker.ai train classifiers to assign identities to individuals in each frame and also extract postural information such head and limb position. In optimized experiments, these trackers can maintain distinct identities over extended periods with minimal human intervention. Other trackers, such as Ctrax ToxTrac, and Tracktor \cite{Branson_High_2009,Rodriguez_ToxId_2017,Sridhar_2019}, track animals by segmenting them from the image background and assign identities by stitching traces together across frames based on changes in position. Although the classification accuracy can be quite high under optimal conditions, these methods generally require human intervention to prevent assignment error from propagating over longer timescales even at low error rates (or they are used for analyses where individual identity is not needed). 

Both approaches to identity tracking can be used to study complex social and individual behaviors, but the computational cost of collision resolution means that tracking is generally performed offline on recorded video data \cite{Liu_A_2018}. Furthermore, the need to record high-quality, high-resolution video data can make it challenging to track animals over long experiments. Some methods of postural segmentation require manual addition of limb markers \cite{Kain_Leg_2013}, splines fit in post-processing \cite{Uhlmann_FlyLimbTracker_2017}, or computationally heavy machine vision in post-processing \cite{Mathis_DeepLabCut_2018,Pereira_Fast_2018,romero-ferrero_2019}. In all cases, the need to separate tracking and recording can be rate-limiting for experiments. Real-time tracking offers the benefits of allowing closed-loop stimulus delivery and a small data footprint due to video data not being retained. In general, real-time tracking methods are less capable of tracking individuals through collisions because they cannot use future information to help resolve ambiguities \cite{Itskovits_A_2017}. For that reason, real-time multiple animal trackers can fall back on spatial segregation of animals to distinguish identities or dispense with identity tracking altogether \cite{Liu_A_2018,Scaplen_Automated_2019}. Some existing real-time trackers can track multiple animals (without maintaining their identity through collisions) in parallel and support a variety of features such as modular arena design, and closed-loop stimulus delivery \cite{Geissmann_Ethoscopes_2017,Straw_Multi_2010,Stowers_Virtual_2017,Chagas_The_2017}.

The tracking algorithms, software interface, hardware configurations, and experimental goals of available trackers vary greatly. Some packages such as Tracktor and FlyWorld use a simple application programming interface (API) and implement tracking through background segmentation and match identities with Hungarian-like Algorithms that minimize frame-to-frame changes in position \cite{Kuhn_The_1955,Rodriguez_ToxId_2017,Liu_A_2018}. Ethoscopes are an integrated hardware and software solution that take advantage of the small size and low cost of microcomputers such as the Raspberry Pi. They support modular arenas and peripheral hardware for stimulus delivery \cite{Geissmann_Ethoscopes_2017} and can be networked and operated through a web-based interface to conduct experiments remotely and at scale. Ethoscopes provide a hardware template and API for integrating peripheral components into behavioral experiments, but the Ethoscope tracker is not currently designed to operate independent of the hardware module. BioTracker offers a graphical user-interface (GUI) that allows the user to select from different tracking algorithms with easily customized tracking parameters or import and use a custom algorithm \cite{Mnck_BioTracker_2018}. 

We wanted a platform that integrated many of the positive features of these trackers into a single software package, while supporting genome-scale screening experiments in a flexible way that would support the needs of labs that study diverse behaviors. We prioritized 1) fast and accurate individual tracking that could be scaled to very large numbers of individuals or experimental groups over very long timescales, 2) flexibility in the user interface that would permit a diversity of organisms, tracking modes, experimental paradigms, and behavioral arenas, 3) integration of peripheral hardware to enable closed-loop sensory and optogenetic stimuli, and 4) a user-friendly interface and data output format. 

We developed MARGO, a MATLAB based tracking suite, with these goals in mind. MARGO can reliably track up to thousands of individuals simultaneously in real-time for days or longer (with limits only set by logistical challenges such as keeping animals fed). MARGO has two tracking modes that allow it to distinguish either individuals or groups of individuals that are spatially segregated. We show that traces acquired in MARGO are comparably accurate to those of other trackers and are robust to noisy images and changing imaging conditions. We also demonstrate that tracking works reliably with nonspecialist equipment (like smart phone cameras). MARGO provides visual feedback on tracking performance that streamlines parameter configuration, making it easy to setup new experiments. 

Additionally, MARGO can control peripheral hardware, enabling closed-loop individual stimulus delivery in high-throughput paradigms. Using adult fruit flies, we demonstrate three closed-loop \cite{heisenberg_wolf_1984} applications in MARGO for delivering individualized stimuli to multiple animals in parallel. First we measured individual phototactic bias in Y-shaped arenas. Second we quantified individual optomotor response in circular arenas. In the third assay, we configured MARGO to deliver optogenetic stimulation in real-time. Though MARGO was developed and tested with adult fruit flies, we show that it can be used to track many organisms such as fruit fly larvae, nematodes, larval zebrafish and bumblebees. We packaged MARGO with an easy-to-use graphical user interface (GUI) and comprehensive documentation to improve the accessibility of the software and offer it as a resource to the ethology community. Though it does not perform visual identity recognition or postural limb tracking, we believe that MARGO can meet the needs of many large behavioral screens, experiments requiring real-time stimulus delivery, and users looking to run rapid pilot experiments with little setup.

% Results and Discussion can be combined.
\subsection*{\textit{MARGO workflow}}

The core experimental workflow of a MARGO experiment (fig. 1A) can be briefly summarized as follows: 1) define spatial regions of interest (ROIs) in which flies will be tracked, 2) construct a background image used to separate foreground and background, 3) compute statistics on the distribution of the foreground pixels under clean tracking conditions to facilitate detection and correction of noisy imaging, 4) perform tracking. We found that constraining the space in which an animal might be located significantly relaxed the computational requirements of multi-animal tracking. Because MARGO is designed for high-throughput experiments, it needs to be convenient to define up to thousands of ROIs. MARGO has two modes for defining ROIs. The first is automated detection that detects and segments regular patterns of high-contrast regions in the image, such as back-lit arenas. The second prompts the user to manually place grids of ROIs of arbitrary size. In practice, we find that ROI definition typically takes a few seconds but can take as long a few minutes.

Following ROI definition, a background image is constructed for each ROI separately. Each background image is computed as the mean or median (as configured by the user) image from a rolling stack of background sample images. Tracking is performed by segmenting binary blobs from a thresholded difference image computed by subtracting each frame from an estimate of the background (fig. 1B). Background subtraction commonly suffers from two issues with opposing solutions. The first is that subtle changes in the background over time introduce error in the difference image, requiring continuous averaging or reacquisition of the background image. The second is that continuous averaging or reacquisition of the background can make inactive animals appear as part of the background rather than foreground, making them undetectable in the thresholded difference image. Constructing the reference for each ROI separately mitigates these concerns by allowing the reference to be constructed in a piece-meal fashion by adding a background sample image only when the animals have moved from the positions they occupied in previous images of the background stack. The time needed to establish a background image depends on the activity level of the animals and the number of images in the reference stack. We typically find that 3-30 seconds are needed to initialize the background image. Once a background image is established, tracking can begin. In each frame, candidate blobs are identified as the blobs that are both 1) between minimum and maximum size threshold and 2) located within the bounds of an ROI. Candidate blobs are subsequently assigned to ROIs by spatial location. Within each ROI, candidate blobs are matched to centroid traces by minimizing the total frame-to-frame changes in position within each ROI.If the number of candidates exceeds the number of traces in a given ROI, only the candidates closest to the last centroid positions of the traces are assigned. If the number of traces exceeds the number of candidates, the candidates are assigned to the closest traces and any remaining traces are assigned no position (i.e. NaN) for that frame.

Degradation of difference image quality over time (due changes in the background, noisy imaging, and physical perturbation of the imaging setup) constitutes a significant barrier to long term tracking \cite{Sridhar_2019}. To address this problem, MARGO continuously monitors the quality of the difference image and updates or reacquires the background image when imaging becomes noisy. We refer to this collective process as noise correction. Prior to tracking, MARGO samples the distribution of the total number of above-threshold pixels under clean imaging conditions to serve as a baseline for comparison. During tracking, the software then continuously calculates that distribution on a rolling basis and reacquires a background image when the rolling sample substantially deviates from the baseline distribution.

\subsection*{\textit{Tracking accuracy and noise robustness}}

We performed a number of experiments and analyses to assess MARGO's robustness to tracking errors and comparability with other trackers. In these experiments, we tracked individual flies, each alone in a circular arena, so that individual identity was assured by spatial segregation.

We assessed the ability of MARGO to handle degradation of the difference image by repeatedly shifting the background image by a small amount in a random direction (2px, 2\% of the arena diameter, and 0.16\% of the width of the image) to mimic situations where an accidental nudge or vibration shifts the arena. MARGO was used to simultaneously record a movie of individual flies walking in circular arenas and track their centroids. These tracks were the ground truth for this misalignment experiment, and background shifting was implemented digitally on the recorded movie. MARGO reliably detects the changes in difference image statistics associated with each of these events and recovers clean tracking by reacquiring the background, typically within 1 second (fig. 2A-B). Forcing reacquisition of the background image has the disadvantage of resetting the reference with a single image, meaning that a normal background image built by median-filtering multiple frames spaced in time cannot be computed immediately (background images made this way have two benefits: lower pixel noise and fewer tracking dead spots because they do not include moving animals). This typically caused a reduction in tracking accuracy that is brief (<2s) and had little effect on the overall correlation of the tracking data to the ground truth (r=0.9998). Indeed, we found a small effect on tracking error (mean 3.07+/- 2.5 pixels, which corresponds to ~20\% of a fly's body length at our typical imaging resolution) even when shifting the background every 2 seconds. In our experimental set-ups, noise-induced background reacquisition was relatively rare, typically occurring fewer than 10 times over the duration of a two hour experiment.

We tested MARGO's sensitivity to video compression by compressing and tracking a video previously captured during a real-time tracking session. The centroid position error of traces acquired from compressed videos were calculated by comparing them to the ground-truth traces acquired on uncompressed images in real time. MARGO showed sub-pixel median tracking error up to 3000-fold compression (fig. 2C). We further tested the robustness of MARGO to noisy imaging by digitally injecting pixel noise (by randomly setting each pixel to True with a fixed probability) into the thresholded difference image of each frame of a video previously acquired and tracked under clean conditions. Noise was added downstream of noise correction and upstream of tracking to simulate tracking under conditions where noise correction is poorly calibrated. We observed sub-pixel median tracking error up to 20\%pixel noise (fig. 2D). In practice, we find it easy to create imaging conditions with noise levels <1\% pixel noise without the use of expensive hardware.

To compare the tracking accuracy of MARGO to a widely used animal tracker, we fed uncompressed video captured during a live tracking session in MARGO into Ctrax \cite{Branson_High_2009} and measured the discrepancy between the two sets of tracks. Overall we found a high degree of agreement between traces acquired in MARGO and Ctrax (fig. 2E-F). We attribute the majority of discrepancies to minor variations in blob size and shape arising from differences in background segmentation. It is worth noting that although Ctrax flagged many frames for manual inspection and resolution, for comparability we opted not to resolve these frames and instead restricted our analysis to the automatically acquired traces. (Ctrax primarily uses these flags to draw user attention to tracking ambiguities through collisions, which did not happen in our experiment because flies were spatially segregated.) Manual inspection of tracked frames with error larger than 1 pixel revealed that most major discrepancies occurred in one of two ways: 1) short periods between the death and birth of two traces on the same animal in Ctrax, or 2) identity swaps in Ctrax between animals in neighboring arenas. These errors may be attributable to our inartful use of Ctrax.

\subsection*{\textit{High-throughput behavioral screens}}

We designed MARGO with high-throughput behavioral screens in mind, with hundreds of experimental groups, each potentially containing hundreds or thousands of animals. Many features in MARGO's GUI have been included to reduce the time needed to establish successful tracking, including automated ROI detection and visualizations of object statistics and the effects of parameters. Configuring tracking for experiments with hundreds of individuals typically took between 2-5 minutes. Additionally, we added the ability to save and load parameter and experimental configurations.

The speed of the tracking algorithm permits the tracking of very large numbers of animals simultaneously in a single field of view (facilitating certain experimental designs, like testing multiple experimental groups simultaneously). To demonstrate MARGO's throughput, we continuously tracked 960 flies at 8Hz for more than 6 days (supplemental video 1). Flies were singly housed in bottomless 96-well plates (fig. 3A) placed on top of food and were imaged by a single overhead camera. The appearance of the arenas changed substantially over 6 days due to evaporation of water from the fly food media, condensation on the well plate lids, and egg laying. Despite these changes, the quality of centroid traces and acquisition rates appeared stable throughout the experiment (fig 3B). The overall activity level of flies decreased over the duration of the experiment (fig 3C). The flies' log-speed distributions generally exhibited two distinct modes: a low mode consistent with frame-to-frame tracking noise and a higher mode consistent with movement of the flies (fig 3D)\cite{berman_choi_bialek_shaevitz_2014,Crall_2016_cockroach}. Individual flies varied in the relative abundance of these two modes. We defined a movement threshold as the local minimum between these two modes and parsed individual speed trajectories into movement bouts by identifying periods of continuous movement above the threshold. Sorting flies by the average length of their movement bouts revealed a trend of increasing mean and magnitude of the higher "movement" mode (fig. 3D), i.e., flies that walked longer tended to walk faster.

To measure MARGO's performance as a function of the number of ROIs, we recorded the mean real-time tracking rate while varying the number of tracked ROIs from a high-resolution (7.4MP) video composed of the same single-arena video repeated 2400 times in a grid. We found that the frame-to-frame latency scaled linearly as a function of the number of ROIs tracked (fig. 3E). On modern computer hardware (intel i7 4.0GHz CPU), we measured tracking rates of 160Hz for a single ROI down to 5Hz for 2400 ROIs. MARGO could plausibly track up to 5000 animals at lower rates (~1 Hz), potentially fast enough for experiments monitoring changes in activity level changes over long timescales, like circadian experiments.

Large behavioral screens can potentially generate hundreds of hours of data on thousands of animals and massive data files even without recording videos. We found that experiments tracking many hundreds of animals over multiple days made raw data files too large to hold in memory on typical computers. We designed a custom data container and an API to easily work with data stored in large binary files. MARGO's raw data API includes methods to batch-process multiple tracking experiments or single datasets too large to hold in memory (see user documentation). 

\subsection*{\textit{Customization and versatility}}

To demonstrate MARGO's ability to prototype experiments without the need for specialized hardware, we ran a minimal tracking experiment using only commonly available materials. Individual fruit flies were placed into the wells of a standard 48 well culture plate. The plate was put in a cardboard box (to reduce reflections) on a sheet of white paper as a high contrast background. Movies were recorded on a 1.3MP smartphone camera using natural room light as illumination and imported into MARGO for tracking (supplemental video 2). Tracks and movement bouts acquired under these conditions showed no apparent differences to those acquired under our normal experimental conditions (custom arenas over diffused LED illuminators in light-sealed imaging boxes). However, we did find that the lower contrast illumination of this setup increased imaging noise and narrowed the range of parameters that worked for segmentation, but had no apparent effect on the accuracy of traces once calibrated.

MARGO was developed for high-throughput ethology in fruit flies, but many small organisms used for high-throughput behavior are more translucent than adult flies. To assess MARGO’s tracking robustness on such organisms, we used MARGO to track videos of larval \emph{Danio rerio}, \emph{Caenorhabditis elegans}, larval \emph{Drosophila} (supplemental videos 3-5), and also bumblebees (\emph{Bombus impatiens}) (supplemental videos 6). As expected, the translucency of these organisms narrowed the functional range of some tracking parameters, but MARGO's real-time tracking feedback made it easy to dial in these parameters. Sample traces acquired from other organisms were qualitatively similar to those acquired with adult flies, suggesting that MARGO works with a variety of organisms.

We gave MARGO a graphical user interface (GUI) to make it accessible to users unfamiliar with MATLAB or programming in general (fig. 1D). We generally find that new users easily learn to use both the core work-flow and parameter customization. The typical setup time of a tracking experiment for trained users ranged between a few seconds (with saved parameter profiles) to a few minutes (under novel imaging conditions). The utility of the GUI extends to customization of analysis, visualization, and input/output sources such as videos, cameras, displays, and COM devices. Descriptions and instructions for these use cases, including defining custom experiments via the API, are available in MARGO's documentation.

\subsection*{\textit{Integrating hardware for closed-loop experiments}}

Real-time tracking allows the delivery of closed-loop stimuli that depend on the behavior of animals. MARGO offers native support for the hardware needed for closed-loop experiments including: cameras for real-time image acquisition, projectors/displays for visual stimuli, and serial COM devices for digital control of other peripheral electronics. COM devices include programmable microcontrollers (like Arduinos) that make it relatively simple to control a wide variety of devices. MARGO was designed to detect and communicate with such COM devices devices to integrate real-time feedback from sensors and coordinate closed-loop control of peripheral hardware. 

We ran experiments with a custom circuit board to measure individual phototactic preference (the "LED Y-maze"). In this assay, individual flies explored symmetrical Y-shaped arenas with LEDs at the end of each arm (fig. 4A-B, supplemental video 7). For all arenas in parallel, real-time tracking detected which arm the fly was in at each frame. At the start of each trial, an LED was randomly turned on in one of the unoccupied arms. Once the fly walked into one of these two new arms, MARGO turned off all the LEDs in that arena. Immediately after these choice events, a new trial was initiated by randomly turning on an LED in one of the now unoccupied arms. This process repeated for each fly independently over two hours, and MARGO recorded which turns were toward a lit LED (positive phototaxis) and which were away (negative phototaxis) (fig. 4C). Tiling many such mazes on a single board yielded the experimental throughput for which MARGO is well-suited. Overall, we recorded choices from over 3,600 individuals, representing more than 830,000 choices in total.

To assess MARGO's capacity to reveal behavioral differences between genotypes, we tested a variety of wild type strains in the LED Y-maze. All strains exhibited a significant average positive phototactic bias (mean phototactic indices ranging from 0.55 to 0.80, p-values$<<$10$^{-6}$ by t-test). In contrast, blind flies (\emph{Norp-A} mutants) and flies under identical circumstances but with unpowered LEDs, showed mean "preferences" indistinguishable from 0.5, consistent with random choices (fig. 4D). The wild type lines tested showed significant variation in population mean (one-way ANOVA; F(6,1943)=118.2, p$<<$10$^{-6}$) and population variability (one-way ANOVA on Levene-transformed data; F(6,1943)=19.29, p$<<$10$^{-6}$).

We collected LED Y-maze data from a single cohort of wild-type (Berlin-K, n=144) flies over the first 8 days post-eclosion to profile phototaxis throughout development (fig. 4E). Flies displayed a significant average negative light bias (0.417, p$<<$10$^{-6}$) on the day of eclosion but transitioned to a positive light bias of 0.663 (p$<<$10$^{-6}$) by 7 days post-eclosion. This assay has structural similarities to an assay we previously used to measure locomotor handedness \cite{Buchanan_Neuronal_2015}, the tendency of individuals to turn left or right when going through the center of the arena. In the LED Y-maze assay, locomotor left-right decisions were made in superposition with light-dark choices. Flies typically make hundreds of choices over the course of an experiment, giving us enough data to examine the turn bias of individuals in all four left-right/light-dark combinations. We divided trials into two groups based on whether the lit LED appeared to the right or left of the choice point. We found that the mean turn bias but not the mean phototactic bias differed between these two conditions (fig. 4F) \cite{Ayroles_Behavioral_2015}. Categorizing trials this way revealed that the rank order of both turn bias and phototactic bias are anti-correlated (r=-0.38 and r=-0.63 respectively) between the two conditions, suggesting that both individual phototactic bias and locomotor handedness bias affect each choice. 

We adapted an optomotor paradigm \cite{Cruz572792} to a high-throughput configuration to test MARGO's ability to deliver a precise closed-loop stimulus with low latency. In this paradigm, an optomotor stimulus consisting of a high-contrast, rotating pinwheel, centered on a fly, is projected on the floor of the arena in which it is walking freely. On average, such optomotor stimuli evoke a turn in the direction of the rotation to stabilize the visual motion \cite{Gtz_Visual_1973}. The center of the pinwheel follows the position of the fly as it moves around the arena so that the only apparent motion of the stimulus is around the fly. Thus, this stimulus is closed-loop with respect to each animal's position and open-loop with respect to its rotation velocity.

To implement this paradigm, we constructed a behavioral platform with a camera and an overhead mounted projector targeting an array of flat circular arenas (fig. 5A). To target a stimulus to a fly based on its coordinates in the tracking camera, MARGO had to learn the mapping of camera coordinates to projector coordinates. We added a feature to locate small dots displayed by the projector with the camera. From the position of these dots in camera coordinates, we constructed a registration mapping from the camera FOV to the projector display field. Using this mapping, we programmed MARGO to use the real-time positions of flies to project pinwheel stimuli independently to 48 freely moving individuals simultaneously (fig. 5B, supplemental video 8). To ensure faithful coordination between the tracking and stimulus, the tracking rate was matched to the refresh rate of the display at 60Hz (which is below the flies' flicker-fusion rate, meaning this stimulus produces beta movement apparent motion \cite{haag_arenz_serbe_gabbiani_borst_2016}; see Discussion). 

While optimizing this assay, we observed that optomotor responses could be reliably elicited, provided individuals were already moving when the pinwheel was initiated. This is consistent with previous observations of optomotor responses depending on arousal state \cite{Zhu_Peripheral_2009,Kim_Fly_2016}. We therefore configured MARGO to stimulate with the pinwheel each fly when: 1) it was moving 2), a minimum inter-trial interval had passed, and 3) it was a minimum distance away from the edge of the arena. The inter-trial interval helped prevent behavioral responses from adapting, and provided a baseline measurement period where no stimulus was present. Minimum distance to the edge ensured that the stimulus occupied a significant portion of the animal's field of view. 

We characterized the optomotor behavior of wild type flies in a two hour experiment with two second pinwheel stimuli and a minimum inter-trial interval of 2s (fig. 5C). In total, over 300,000 trials were recorded from more than 1,800 flies, assayed in groups of up to 48 flies simultaneously. For each fly, we calculated an optomotor index \cite{Seelig_Two_2010} as the fraction (normalized to [-1,1]) of body angle change that occurred in the same direction as the stimulus rotation over the duration of the stimulus. On average, flies displayed reliable optomotor responses (mean index = 0.358, p$<<$10$^{-6}$) when stimulated with high-contrast pinwheels (fig. 5D). We observed significant individual variation in optomotor index (fig. 5E) as well as the number of trials each fly experienced, reflecting individual variation in the fraction of time walking. 

To characterize the psychometric properties of this behavior, we randomly varied pinwheel contrast, angular velocity, and spatial frequency simultaneously on a trial-by-trial basis. Mean optomotor indices increased with pinwheel contrast, plateauing over much of the dynamic range of the projector, starting around 25\% contrast (fig. 5 F). Similarly, optomotor indices increased with both stimulus spatial frequency and angular speed, peaking at 0.18 cycles/degree and 360 degrees/s respectively (fig. 5 G). The population mean optomotor index reversed at high combined values of spatial frequency and angular speed due to the apparent reversal of the stimulus at frequencies higher than the refresh rate of the projector.

\subsection*{\textit{High-throughput optogenetic experiments}}

To test the versatility of MARGO, we used its API to implement high-throughput closed-loop optogenetic experiments using a digital projector to target individual flies expressing CsChrimson \cite{griebel_2014,Klapoetke_Independent_2014} with flashing red light contingent on their behavior (fig. 6). We used a commericial Optoma S310e DLP projector which, when displaying red light ([255 0 0] RGB code), had a spectral range of 570 nm to 720 nm with a peak at 595 nm. Light stimulation frequency was set to the projector refresh rate (60Hz) and its intensity to the maximum, if not otherwise specified.

As a first experiment, we tracked the flies in a Y-Maze shaped like that in fig. 4A, but with no LEDs. Whenever a fly entered a designated arm, MARGO projected red light on it. Flies expressed CsChrimson in bitter-taste receptor neurons using the driver \textit{Gr66a-GAL4}. MARGO recorded the fractional time spent in the lit arm (occupancy) and the number of entries into the lit arm (entries). We observed a modest increase in the aversive effects of optogenetic stimulation (reduced occupancy and entries) with light intensity (fig. 6A.1), whereas increasing stimulation frequency did not elicit any obvious change in aversion (fig. 6A.2). To test the robustness of the experiment to changes in the fictive conditioning stimulus, and to exclude the effects of visual cues, we expressed CsChrimson in heat sensitive neurons targeted by \textit{Gr28bd+TrpA1-GAL4} in \textit{norpA\textsuperscript{P24}} blind flies. This experiment is conceptually analogous to spatial learning in the heat-box, where flies are trained to avoid one side of a dark, heatable chamber \cite{wustmann_rein_wolf_heisenberg_1996,wustmann_heisenberg_1997,diegelmann_2006,ostrowski_kahsai_kramer_knutson_zars_2015,putz_2002,sitaraman_zars_zars_2007,sitaraman_zars_zars_2010,zars_zars_2006}. While blindness only marginally affected the time spent in the lit arm (the blind flies with Chrimson driven in heat-sensitive neurons still avoided occupying the lit arm at similar rates to seeing flies with Chrimson in bitter-sensitive neurons), the reduction in entries into the lit arm, observed in the seeing flies, was abolished (fig. 6A.3). These results suggest that vision is a key sensory modality informing the decision to enter an arm, but not for the decision of how much time to spend in an arm, once entered.

Analogous to a different heat-box experiment \cite{YANG2013799}, optogenetic stimulation was made contingent on locomotor speed rather than position. In the same circular arenas as the optomotor experiments above (fig. 5A), the red light was switched on under two distinct conditions enforced in separate experimental blocks: 1) whenever the walking speed of the flies exceeded a threshold of 6.8 mm/s and 2) whenever the walking speed fell below that same threshold. The overall 64 minute experimental protocol consisted of 8 periods of 8 minutes each. The periods alternated between a baseline period, where the light was permanently switched off, and the two reinforcement periods where the light was contingent on either fast walking or slow walking/resting, respectively (fig. 6B.1-2). As in the heat-box experiments, flies increased their walking speed when punished for walking too slowly. However, punishing fast walking failed to significantly decrease walking speed. Reminiscent of the induction of 'learned helplessness' in yoked control animals in the heat-box \cite{YANG2013799}, flies trained with these conflicting schedules of punishment, significantly reduced their walking speed in the baseline periods without optogenetic stimulation, in comparison to control animals which did not express any CsChrimson (fig. 6B.3).

\section*{Discussion}
We developed MARGO as a platform for a wide variety of behavioral paradigms and organisms, all at high throughput for large-scale experiments (like genetic screens, measuring individuality and characterizing psychometric response curves). MARGO's tracking algorithm, interface, and data footprint are lightweight, making it perform well in applications like real-time centroid tracking. Conversely, it is not made for harder computational tasks like maintaining the identity of multiple animals in the same compartment. But the ability to rapidly define ROIs, and track individuals in them, enables MARGO to easily coordinate low-latency, closed-loop stimulation for psychometric and optogenetic experiments. Furthermore, by packaging MARGO in a GUI and thoroughly documenting its usage and API, we hope to make it accessible both to new users with little programming experience and advanced users developing custom experimental paradigms.

When ROI boundaries are drawn along physical barriers, individual identities can be maintained indefinitely through ROI identity, thus removing the requirement for human supervision and intervention. We found that insisting on spatial segregation ultimately relaxes the computational requirements enough that thousands of individuals can be tracked in real time. In the future, real-time tracking that maintains individual identity without physical barriers may be possible, perhaps as an extension of current methodologies that exploit neural networks to track individuals offline \cite{romero-ferrero_2019,schneider_2018}. MARGO's interface assists in the automated definition of up to thousands of ROIs. An ROI-based architecture can also be used to distinguish groups rather than individual identities by separating groups into distinct arenas. This configuration therefore allows multiple groups, as well as individuals, to be tested in parallel.

Long-term automated behavioral measurement has great potential in the fields of sleep, circadian rhythms, pharmacology, and aging, among others. MARGO offers many features useful for activity measurement over long timescales, including rapid experimental setup, small data footprint, and built-in utilities for handling large data sets. For example, over a week we tracked the behavior of 960 flies simultaneously as they walked in the wells of custom 96-well plates (fig. 3). Such throughput can be applied to comparisons among individuals, genotypes, or treatment groups. 

With built-in hardware support for cameras, displays, and peripheral electronics, MARGO enables open- and closed-loop stimulus-evoked ethology on a large scale. Built-in features supporting projector displays, like camera-projector registration, facilitate a wide variety of visual and optogenetic experiments (figs. 5, 6). Native detection and communication with serial COM devices further extends these capabilities by providing a generic interface for a wide variety of peripheral devices, such as the LED controllers we used for the LED Y-maze (fig. 4). Taken together, MARGO is a multi-purpose platform for coordinating hardware inputs and outputs for high-throughput ethology. 

Between our two closed-loop visual stimulation experiments (LED Y-maze and optomotor assay), we screened nearly 5,000 animals over hundreds of thousands of trials, allowing the precise characterization of both individual- and population-level behavior. With the experiments themselves representing less than a week of testing, these platforms could be used for large behavioral screens of hundreds of strains. In the LED Y-maze, we showed that individuals displayed idiosyncratic biases in both phototactic preference and locomotor handedness simultaneously, as observed previously in separate assays \cite{Kain_Phototactic_2012,Buchanan_Neuronal_2015}. The wild-type fly lines we screened displayed population level differences in both mean preference and variability in phototactic bias \cite{Ayroles_Behavioral_2015}. Furthermore, the mean of one strain (Berlin-K) shifted from negative to positive over the first week post-eclosion, as was reported previously \cite{Chiang_Tactic_1963}. Interestingly, we observed that flies with a high right-turn probability were more likely to turn toward the light when it was to the right of the choice point and that the opposite was true of flies with a high left-turn probability. We observed a similar but stronger effect of phototactic bias on locomotor handedness (e.g., flies with a high phototactic bias were more likely to turn toward the right when the light was on the right). Together these results demonstrate measurable effects of phototactic bias and handedness in a task that probes both simultaneously. Thus, we found that both individual light and handedness biases influence light/turn behavior on a choice-by-choice basis. As responses to light are ethologically relevant \cite{Kain_Variability_2015}, the interplay of individual behavioral biases may have fitness consequences for wild flies.

In the optomotor experiment, we demonstrated that, using closed-loop stimuli delivered from a projector, MARGO can quantify individual optomotor responses of dozens of flies simultaneously. Consistent with previous findings \cite{Zhu_Peripheral_2009,Kim_Fly_2016}, we saw that stationary flies did not exhibit strong optomotor responses, consistent with the idea that this reflexive behavior may be state-dependent \cite{Rosner_Behavioural_2009, Rien_Octopaminergic_2013, Chiappe_Walking_2010, Maimon_Active_2010,gorostiza_2018}. While all animals tested exhibited the optomotor response to some degree, we observed a broad distribution of individual optomotor indices, suggesting that individuals respond idiosyncratically to the same stimulus, as has been found previously in other spontaneous and stimulus-evoked behaviors \cite{Kain_Phototactic_2012,Kain_Leg_2013,Kain_Variability_2015,Buchanan_Neuronal_2015,Todd_Systematic_2017}. We suspect that the success of this assay may be partially due to tightly centering the pinwheel centered on the animal as it moves, which is possible because of MARGO's low latency.

Our optogenetic experiments provide a proof of principle that high-throughput closed-loop manipulation of neural activity is feasible (fig. 6). Using different driver lines to activate neurons under both spatial (fig. 6A) and locomotor (fig. 6B) contingencies, optogenetic stimulation reliably altered fly behavior in the expected directions. These experiments also revealed that flies use visual elements of the projector rig to orient when the stimulus was nominally off, and that optogenetic punishment can induce learning effects outlasting the stimulation itself. These results also remind us about a general limitation of studying freely moving animals: the large number of degrees of freedom that such behavior enables can make it difficult to causally relate biological manipulations to specific mechanisms. For instance, without prior knowledge of the function of the optogenetically targeted neurons, it would not have been immediately clear if our manipulation affected reinforcing neurons or neurons involved in motor control, which could also lead to altered occupancy of the lit arm in the Y-Maze. Likewise, a screen for neurons that are required for non-random entry into optogenetically-reinforced arms of the Y-maze would yield blind flies, as the flies in our assay apparently use visual cues to identify which arms are reinforced before entering them.

Behavioral experiments are frequently more complex than tracking objects in a dish. Such experiments could require complex arena geometries, data streams from external sensors, control of peripheral hardware, and access to measurements of behavior in real time. MARGO can manage these features, making it well-suited to implementing new behavioral paradigms. Specifically, MARGO can automatically generate templates for new experiments with custom inputs and outputs within the GUI. We have also included a tutorial for defining custom experiments in MARGO's documentation. In practice, we find that new experiments can typically be defined in one or two custom functions, given familiarity with the API.

Animal tracking platforms are evolving to meet the diverse needs of the ethology, neuroscience and behavioral genetics communities. See Table 1 for a comparison of features of several contemporary tracking programs. Trackers can be broadly described as falling into one of two categories: 1) real-time trackers \cite{Fry_TrackFly_2008,Straw_Multi_2010,Chagas_The_2017,Geissmann_Ethoscopes_2017,Mnck_BioTracker_2018,Scaplen_Automated_2019} with potential for high throughput and hardware control and 2) offline trackers \cite{Branson_High_2009,Prez-Escudero_idTracker_2014,Eyjolfsdottir_Detecting_2014,Rodriguez_ToxId_2017,Sridhar_2019,romero-ferrero_2019} with the potential to maintain individual identities (without using spatial segregation) and/or track body parts. Hardware integration is a natural extension of real-time trackers since many stimulus paradigms are contingent on behavior. While trackers in the second category are currently unsuitable for real-time applications, they offer the notable benefits of being able to study fine-scale postural and social behaviors. The ability to record video in parallel with tracking and peripheral hardware control means MARGO can be used upstream of offline trackers, making it possible to analyze social dynamics or postural features in response to closed-loop stimuli. Among this array of options, MARGO is optimized for the throughput characteristic of \textit{Drosophila} and other genetic model organisms like \textit{C. elegans}. MARGO has the flexibility to accommodate the experimental diversity of techniques in neuroethology. Thus, we envision MARGO's niche as a versatile platform for experiments operating at high throughput to measure individual behavior and deliver closed-loop sensory and optogenetic stimuli.

\section*{Methods}

 \subsection*{Repositories}
MARGO's code is available in the \href{https://github.com/de-Bivort-Lab/margo}{MARGO repository} on github. All behavioral data is available on \href{https://zenodo.org/record/2596143#.XI2maRNKiRc}{Zenodo}. Instrument schematics are available on github at: \href{https://github.com/de-Bivort-Lab/dblab-schematics}{de Bivort Lab schematics repository}.

 \subsection*{Software}
 
The MARGO GUI, tracking algorithm, and all analysis software were written in MATLAB (The Mathworks, Inc, Natick, MA). Detailed descriptions of the functions and use of the MARGO GUI, ROI detection, background referencing, tracking implementation, noise correction, and data output are available in MARGO's \href{https://github.com/de-Bivort-Lab/margo/wiki}{documentation}. Optomotor stimuli were crafted and displayed using the Psychtoolbox-3 for MATLAB. Software for control of all custom electronic hardware was written in C using Arduino libraries.

 \subsection*{Organism genotypes and rearing}
 
Unless otherwise specified, the genotype of all fruit flies tested was a strain of Berlin-K that we inbred for 13 generations prior to these experiments. Gr66a-G4 (from the G. Turner lab), \textit{norpA\textsuperscript{P24}} (from the M. Heisenberg lab), TrpA1-G4 (FlyBase ID: 27593), Gr28bd-G4 (FlyBase ID: 57620), UAS-20xCsChrimson (FlyBase ID: 55135) were the lines used in the optogenetic experiments. Tracking experiments were performed with mixed sex flies 3-5 days post-eclosion unless otherwise noted. Flies were raised on standard conrmeal/dextrose formula media (Harvard Fly Core Facility) under 12 h/12 h light and dark cycle in an incubator at 25°C and 40\% humidity. Animals were imaged and singly-housed on food in modified 96 well plates (Fly Plates, FlySorter LLC) for all multi-day tracking experiments. \textit{C.Elegans} were housed in a custom platform on agarose media and were composed of multiple strains as described in the WorMotel publication \cite{Churgin_Longitudinal_2017}. \textit{Drosophila} larvae CantonS on 2\% agarose media mixed with fructose in a gradient (0-300mM) along one axis. Larval zebrafish were \textit{HC:GCaMP6s}.

 \subsection*{Behavioral acquisition}

All real-time tracking images were captured with USB or GigE cameras from Point Grey (Firefly MV 13SC2 and BFLY-PGE-12A2M-CS). Images for the minimal hardware setup were acquired with an iPhone 5s 1.3MP camera. Cameras used in real-time experiments were fitted with a long-pass 87 Kodak Wratten infrared filter with a cutoff frequency of 750nm and illuminated with infrared LEDs centered at 940nm (Knema LLC, Shreveport, LA). Acquisition rates varied by experiment between 5.0 fps for LED Y-maze and 60.0 fps for optomotor response. Flies were imaged at spatial resolutions ranging between 1-4 pixels per mm and we generally found tracking to be stable at 10 pixels per animal and above. Offline video tracking was performed on 1000x compressed AVI video files unless otherwise specified. Tracking and imaging was conducted in Windows 10 on computers with CPUs ranging from intel i3 3.1GHz to intel i7 4.0GHz. 

 \subsection*{Behavioral instruments}

Unless otherwise specified, all tracking was conducted in custom imaging boxes constructed with laser-cut acrylic and aluminum rails. Schematics of custom behavioral arenas and behavioral boxes were designed in AutoCAD. Arena parts were laser-cut from black and clear acrylic and joined with Plastruct plastic weld. Schematics for behavioral boxes can be found on the \href{https://github.com/de-Bivort-Lab/dblab-schematics}{de Bivort Lab schematics repository}. Illumination was provided by dual-channel white and infrared LED array panels mounted at the base (Part\# BK3301, Knema LLC, Shreveport, LA). Adjacent pairs of white and infrared LEDs were arrayed in a 14x14 grid spaced 2.2cm apart. White and infrared LEDs were wired for independent control by MOSFET transistors and a Teensy 3.2 microcontroller. Two sand-blasted clear acrylic diffusers were placed in between the illuminator and the behavioral arena for smooth backlighting. Additional tracking was performed in standard 48 multi-well culture plates and individual fly storage units (FlyPlates) from FlySorter LLC. Additional details on the behavioral platforms used here are available in the MARGO documentation.

\subsection*{Experimental procedures}

Tracking experiments were conducted between 10AM and 6PM. We saw no time-of-day significant effects on individual behavioral measures from the optomotor and LED Y-maze assays. Flies were anesthetized either on ice or CO$_{2}$ and manually loaded into behavioral arenas with an aspirator. Behavioral modules were loaded into tracking boxes and allowed a minimum post-anesthesia recovery period of 20 minutes before tracking. Unless otherwise specified, animals were tracked for 2 hours in an environmental room at 23 C and 40\% humidity. Following tracking, flies were returned to individual storage plates where they were housed for further experiments as needed. For the optogenetic experiments, flies were tested for 20 min in the Y-mazes or 64 min in the circular arenas.

\subsection*{Data and statistics}

Unless noted, all reported error bars are 95\% confidence intervals computed by bootstrap resampling. Data processing and calculation of behavioral metrics was conducted automatically by MARGO either in real time, or after experiments. 1000 bootstrap replicates were averaged to estimate null distributions and confidence intervals. Reported p-values for phototaxis, optogenetic closed-loop experiments and optomotor behavior were unadjusted for multiple comparisons and were calculated via two-tailed t-tests. Critical values were adjusted for multiple comparisons via Bonferroni correction.

% insert chapter title on its own page
\clearpage
\begin{center}
    \Large\textbf{Chapter 3}
    \thispagestyle{empty}       % set page numbering style to blank
    \clearpage
\end{center}

\section*{Chapter 2 Title}

\end{document}